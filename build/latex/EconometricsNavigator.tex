%% Generated by Sphinx.
\def\sphinxdocclass{report}
\documentclass[letterpaper,10pt,english]{sphinxmanual}
\ifdefined\pdfpxdimen
   \let\sphinxpxdimen\pdfpxdimen\else\newdimen\sphinxpxdimen
\fi \sphinxpxdimen=.75bp\relax

\usepackage[utf8]{inputenc}
\ifdefined\DeclareUnicodeCharacter
 \ifdefined\DeclareUnicodeCharacterAsOptional
  \DeclareUnicodeCharacter{"00A0}{\nobreakspace}
  \DeclareUnicodeCharacter{"2500}{\sphinxunichar{2500}}
  \DeclareUnicodeCharacter{"2502}{\sphinxunichar{2502}}
  \DeclareUnicodeCharacter{"2514}{\sphinxunichar{2514}}
  \DeclareUnicodeCharacter{"251C}{\sphinxunichar{251C}}
  \DeclareUnicodeCharacter{"2572}{\textbackslash}
 \else
  \DeclareUnicodeCharacter{00A0}{\nobreakspace}
  \DeclareUnicodeCharacter{2500}{\sphinxunichar{2500}}
  \DeclareUnicodeCharacter{2502}{\sphinxunichar{2502}}
  \DeclareUnicodeCharacter{2514}{\sphinxunichar{2514}}
  \DeclareUnicodeCharacter{251C}{\sphinxunichar{251C}}
  \DeclareUnicodeCharacter{2572}{\textbackslash}
 \fi
\fi
\usepackage{cmap}
\usepackage[T1]{fontenc}
\usepackage{amsmath,amssymb,amstext}
\usepackage{babel}
\usepackage{times}
\usepackage[Bjarne]{fncychap}
\usepackage[dontkeepoldnames]{sphinx}

\usepackage{geometry}

% Include hyperref last.
\usepackage{hyperref}
% Fix anchor placement for figures with captions.
\usepackage{hypcap}% it must be loaded after hyperref.
% Set up styles of URL: it should be placed after hyperref.
\urlstyle{same}

\addto\captionsenglish{\renewcommand{\figurename}{Fig.}}
\addto\captionsenglish{\renewcommand{\tablename}{Table}}
\addto\captionsenglish{\renewcommand{\literalblockname}{Listing}}

\addto\captionsenglish{\renewcommand{\literalblockcontinuedname}{continued from previous page}}
\addto\captionsenglish{\renewcommand{\literalblockcontinuesname}{continues on next page}}

\addto\extrasenglish{\def\pageautorefname{page}}

\setcounter{tocdepth}{0}



\title{Econometrics Navigator Documentation}
\date{Nov 20, 2018}
\release{0.0.1}
\author{Evgeny Pogrebnyak}
\newcommand{\sphinxlogo}{\vbox{}}
\renewcommand{\releasename}{Release}
\makeindex

\begin{document}

\maketitle
\sphinxtableofcontents
\phantomsection\label{\detokenize{index::doc}}



\chapter{Foreword}
\label{\detokenize{forward::doc}}\label{\detokenize{forward:welcome-to-econometrics-navigator}}\label{\detokenize{forward:foreword}}

\section{Motivation}
\label{\detokenize{forward:motivation}}
Back in graduate school I was perfectly happy with Greene, Dougherty,
EViews and Matlab. Now I code in Python more often than in R, and I feel
amount of information about econometrics is immense to a degree
it erodes confidence.

To keep on coding and be sure I do econometrics right I wanted:
\begin{itemize}
\item {} 
a guide that helps refresh easier topics and navigate through more difficult ones

\item {} 
complex things told simply, less jargon and more intuitive appeal

\item {} 
a good list of worthwhile articles to invest time replicating them

\item {} 
credible references to share with students

\end{itemize}

A single guide of the type does not seem to exist, so I’m writing one.


\section{What is happening now?}
\label{\detokenize{forward:what-is-happening-now}}
\sphinxstyleemphasis{November 11, 2018}
\begin{itemize}
\item {} 
updating \sphinxhref{tweets.md}{Twitter} posts page, plenty of information
with a personal appeal

\item {} 
OLS, PCA and Bayesian inference seem to be first topics in
‘Concepts and techniques’ section

\end{itemize}


\section{Contacts}
\label{\detokenize{forward:contacts}}
Feel free to contact me on \sphinxhref{https://twitter.com/PogrebnyakE}{@PogrebnyakE}
about the Navigator draft.


\chapter{1. Interviews}
\label{\detokenize{interviews::doc}}\label{\detokenize{interviews:interviews}}
In this chapter I propose to put down several interviews about
current state of econometrics, problems in teaching, econometrics in
Russia vs abroad, replication, suggestions about textbooks and
learning resources and other personal experiences related to econometrics.

The interviews may be a transcript or a video.

Suggested speakers:
\begin{itemize}
\item {} 
Evgeniy Lukash

\end{itemize}


\chapter{2. Concepts and techniques}
\label{\detokenize{topics::doc}}\label{\detokenize{topics:concepts-and-techniques}}

\section{2.1. Concepts}
\label{\detokenize{topics:concepts}}\begin{itemize}
\item {} 
\sphinxhref{http://onlinestatbook.com/2/probability/bayes\_demo.html}{Bayes theorem}

\item {} 
bias-variance tradeoff

\item {} 
indentification

\item {} 
inference

\item {} 
overfitting

\item {} 
replication, replicability

\item {} 
\sphinxhref{https://mpra.ub.uni-muenchen.de/59008/}{spurious regression}

\end{itemize}


\subsection{Bootstrap}
\label{\detokenize{concepts/bootstrap::doc}}\label{\detokenize{concepts/bootstrap:bootstrap}}
Bootstrapping is a non-parametric method to construct empiric distributions of various
statistics (mean, confidence intervals, deviation, etc) by repreated
sampling from an observed dataset.

A little magic is \sphinxhref{https://stats.stackexchange.com/questions/26088/explaining-to-laypeople-why-bootstrapping-works}{why exactly} taking random samples
like \sphinxcode{{[}1,1,2{]}}, \sphinxcode{{[}3,2,2{]}}, \sphinxcode{{[}1,2,3{]}}, etc is a good idea to approximate
statistic properties of a dataset \sphinxcode{{[}1,2,3{]}}.

Bootstrap originally proposed by \sphinxhref{http://jeti.uni-freiburg.de/studenten\_seminar/stud\_sem\_SS\_09/EfronBootstrap.pdf}{Bradley Efron in 1979}.
It is a very popular approach because it works
well on complex models. See overviews by \sphinxhref{https://www.sciencedirect.com/science/article/pii/S157344120105005X}{Horowitz 2001 in Chapter 52 of Handbook of Econometrics} and by \sphinxhref{https://core.ac.uk/download/pdf/6494253.pdf}{MacKinnon 2006}.
\sphinxhref{https://www.statisticshowto.datasciencecentral.com/bootstrap-sample/}{\sphinxincludegraphics{{bootstrap-sample}.png}}

\subsubsection{Toy example}
\label{\detokenize{concepts/bootstrap:toy-example}}
Code source: \sphinxurl{https://math.mit.edu/~dav/05.dir/class24-empiricalbootstrap.r}.
Commented in \sphinxhref{https://ocw.mit.edu/courses/mathematics/18-05-introduction-to-probability-and-statistics-spring-2014/readings/MIT18\_05S14\_Reading24.pdf}{Bootstrap confidence intervals by Jeremy Orloff and
Jonathan Bloom, pp. 4-6}

\begin{sphinxVerbatim}[commandchars=\\\{\}]
\PYG{c+c1}{\PYGZsh{} Example. Empirical bootstrap confidence interval for the mean.}
\PYG{c+c1}{\PYGZsh{} Data for the example in class24\PYGZhy{}prep}
\PYG{k+kp}{cat}\PYG{p}{(}\PYG{l+s}{\PYGZdq{}}\PYG{l+s}{Example. Empirical boostrap confidence interval for the mean.\PYGZdq{}}\PYG{p}{,}\PYG{l+s}{\PYGZsq{}}\PYG{l+s}{\PYGZbs{}n\PYGZsq{}}\PYG{p}{)}
x \PYG{o}{=} \PYG{k+kt}{c}\PYG{p}{(}\PYG{l+m}{30}\PYG{p}{,}\PYG{l+m}{37}\PYG{p}{,}\PYG{l+m}{36}\PYG{p}{,}\PYG{l+m}{43}\PYG{p}{,}\PYG{l+m}{42}\PYG{p}{,}\PYG{l+m}{43}\PYG{p}{,}\PYG{l+m}{43}\PYG{p}{,}\PYG{l+m}{46}\PYG{p}{,}\PYG{l+m}{41}\PYG{p}{,}\PYG{l+m}{42}\PYG{p}{)}
n \PYG{o}{=} \PYG{k+kp}{length}\PYG{p}{(}x\PYG{p}{)}
\PYG{k+kp}{set.seed}\PYG{p}{(}\PYG{l+m}{1}\PYG{p}{)}  \PYG{c+c1}{\PYGZsh{} for repeatability}

\PYG{c+c1}{\PYGZsh{} sample mean}
xbar \PYG{o}{=} \PYG{k+kp}{mean}\PYG{p}{(}x\PYG{p}{)}
\PYG{k+kp}{cat}\PYG{p}{(}\PYG{l+s}{\PYGZdq{}}\PYG{l+s}{data mean = \PYGZdq{}}\PYG{p}{,}xbar\PYG{p}{,}\PYG{l+s}{\PYGZsq{}}\PYG{l+s}{\PYGZbs{}n\PYGZsq{}}\PYG{p}{)}
nboot \PYG{o}{=} \PYG{l+m}{20}
\PYG{c+c1}{\PYGZsh{} Generate 20 bootstrap samples, i.e. an n x 20 array of}
\PYG{c+c1}{\PYGZsh{} random resamples from x.}
tmpdata \PYG{o}{=} \PYG{k+kp}{sample}\PYG{p}{(}x\PYG{p}{,}n\PYG{o}{*}nboot\PYG{p}{,} replace\PYG{o}{=}\PYG{k+kc}{TRUE}\PYG{p}{)}
bootstrapsample \PYG{o}{=} \PYG{k+kt}{matrix}\PYG{p}{(}tmpdata\PYG{p}{,} nrow\PYG{o}{=}n\PYG{p}{,} ncol\PYG{o}{=}nboot\PYG{p}{)}

\PYG{c+c1}{\PYGZsh{} Compute the means xbar*}
xbarstar \PYG{o}{=} \PYG{k+kp}{colMeans}\PYG{p}{(}bootstrapsample\PYG{p}{)}

\PYG{c+c1}{\PYGZsh{} Compute delta* for each bootstrap sample}
deltastar \PYG{o}{=} xbarstar \PYG{o}{\PYGZhy{}} xbar

\PYG{c+c1}{\PYGZsh{} Find the 0.1 and 0.9 quantiles for deltastar}
d \PYG{o}{=} quantile\PYG{p}{(}deltastar\PYG{p}{,}\PYG{k+kt}{c}\PYG{p}{(}\PYG{l+m}{0.1}\PYG{p}{,}\PYG{l+m}{0.9}\PYG{p}{)}\PYG{p}{)}

\PYG{c+c1}{\PYGZsh{} Calculate the 80\PYGZbs{}\PYGZpc{} confidence interval for the mean.}
ci \PYG{o}{=} xbar \PYG{o}{\PYGZhy{}} \PYG{k+kt}{c}\PYG{p}{(}d\PYG{p}{[}\PYG{l+m}{2}\PYG{p}{]}\PYG{p}{,}d\PYG{p}{[}\PYG{l+m}{1}\PYG{p}{]}\PYG{p}{)}
\PYG{k+kp}{cat}\PYG{p}{(}\PYG{l+s}{\PYGZsq{}}\PYG{l+s}{Bootstrap confidence interval: [\PYGZsq{}}\PYG{p}{,}ci\PYG{p}{,}\PYG{l+s}{\PYGZsq{}}\PYG{l+s}{]\PYGZsq{}}\PYG{p}{,}\PYG{l+s}{\PYGZsq{}}\PYG{l+s}{\PYGZbs{}n\PYGZsq{}}\PYG{p}{)}
\end{sphinxVerbatim}


\paragraph{Bootstrap do’s and don’ts by Anna Mikusheva}
\label{\detokenize{concepts/bootstrap:bootstrap-dos-and-donts-by-anna-mikusheva}}\begin{itemize}
\item {} 
If you have a pivotal statistic,  bootstrap can give a refinement.  So, if you have choice
of statistics, bootstrap a pivotal one.

\item {} 
Bootstrap may fix a finite-sample bias, but cannot help if you have inconsistent estimator.

\item {} 
In  general,  if  something  does  not  work  with  traditional  asymptotics,  the
bootstrap  cannot  fix  your problem. For example, if we have an inconsistent estimate, the
bootstrap bias correction does not fix anything.

\item {} 
Bootstrap could not fix the following problems: weak instruments, parameter on a boundary,
unit root, persistent regressors.

\item {} 
Bootstrap requires re-centering (the null hypothesis to be true).

\end{itemize}

Source: \sphinxhref{https://ocw.mit.edu/courses/economics/14-384-time-series-analysis-fall-2013/lecture-notes/MIT14\_384F13\_lec9.pdf}{MIT lecture notes}


\subsubsection{More links:}
\label{\detokenize{concepts/bootstrap:more-links}}\begin{itemize}
\item {} 
\sphinxurl{https://core.ac.uk/download/pdf/6494253.pdf}

\item {} 
\sphinxurl{https://github.com/wmutschl/GMMIndirectInferenceBootstrap}

\item {} 
\sphinxurl{https://www.schmidheiny.name/teaching/bootstrap2up.pdf}

\item {} 
\sphinxurl{http://rosetta.ahmedmoustafa.io/bootstrap/}

\item {} 
\sphinxurl{http://www.cs.cornell.edu/courses/cs1380/2018sp/textbook/chapters/11/2/bootstrap.html}

\end{itemize}


\subsubsection{Editor notes}
\label{\detokenize{concepts/bootstrap:editor-notes}}
\sphinxhref{https://ru.wikipedia.org/wiki/\%D0\%91\%D1\%83\%D1\%82\%D1\%81\%D1\%82\%D1\%80\%D1\%8D\%D0\%BF\_(\%D1\%81\%D1\%82\%D0\%B0\%D1\%82\%D0\%B8\%D1\%81\%D1\%82\%D0\%B8\%D0\%BA\%D0\%B0)}{Russian article} in Wikipedia on bootstrap is much more concise and understandable
than \sphinxhref{https://ru.wikipedia.org/wiki/\%D0\%91\%D1\%83\%D1\%82\%D1\%81\%D1\%82\%D1\%80\%D1\%8D\%D0\%BF\_(\%D1\%81\%D1\%82\%D0\%B0\%D1\%82\%D0\%B8\%D1\%81\%D1\%82\%D0\%B8\%D0\%BA\%D0\%B0)}{English one}.


\subsection{Causation, causality}
\label{\detokenize{concepts/causation::doc}}\label{\detokenize{concepts/causation:causation-causality}}\begin{quote}

Correlation is not causation.
\end{quote}
\begin{itemize}
\item {} 
\sphinxhref{https://chrisaulddotcom.wordpress.com/2013/10/08/remarks-on-chen-and-pearl-on-causality-in-econometrics-textbooks/}{causality (not ‘casuality’)}

\end{itemize}


\subsubsection{Book of Why}
\label{\detokenize{concepts/causation:book-of-why}}



\subsubsection{History}
\label{\detokenize{concepts/causation:history}}\begin{itemize}
\item {} 
\sphinxhref{ucla.in/2mhxKdO}{Pearl, J. (2014). TRYGVE HAAVELMO AND THE EMERGENCE OF CAUSAL CALCULUS. Econometric Theory, 31(1), 152\textendash{}179. https://doi.org/10.1017/s0266466614000231}

\end{itemize}


\section{2.2. Techniques}
\label{\detokenize{topics:techniques}}

\subsection{Ordinary least squares, OLS}
\label{\detokenize{techniques/ols::doc}}\label{\detokenize{techniques/ols:ordinary-least-squares-ols}}
OLS is at the core of econometrics curriculum, it is easily derived and
highly practical to familiarise a learner with regression possibilites
and limitations.

The usual way to teach OLS is to present assumptions and show how to deal
with their violations as indicated below in a review chart from Kennedy’s
textbook.

\noindent\sphinxincludegraphics{{peter_kennedy_on_ols}.png}

Math:

\(Y = \beta X + \epsilon\)

Assumptions: \(\epsilon\) is normally distributed

Common steps:
\begin{enumerate}
\item {} 
specify model: select explanatory variables, transform them if needed

\item {} 
estimate coefficients

\item {} 
elaborate on model quality (the hardest part)

\item {} 
go to 1 if needed

\item {} 
know what model \sphinxstyleemphasis{does not} show (next hardeer part)

\end{enumerate}

What may go wrong:
\begin{itemize}
\item {} 
residuals are not random

\item {} 
variables are cointegrated

\item {} 
multicollinearity in regressors

\item {} 
residuals depend on x (heteroscedasticity)

\item {} 
inference is not causality

\item {} 
wrong signs, insignificant coefficients

\item {} 
variable normalisation was not described

\end{itemize}

Discussion:
\begin{itemize}
\item {} 
why sum of squares as a loss function?

\item {} 
connections to bayesian estimation

\end{itemize}

Replication examples: -
\sphinxurl{https://www.kaggle.com/nicapotato/in-depth-simple-linear-regression}

Links:
\begin{itemize}
\item {} 
\sphinxurl{https://blog.minitab.com/blog/how-to-choose-the-best-regression-model}

\item {} 
\sphinxurl{https://towardsdatascience.com/simple-linear-regression-2421076a5892}

\item {} 
\sphinxurl{https://papers.ssrn.com/sol3/papers.cfm?abstract\_id=1736184}

\item {} 
\sphinxurl{https://cfss.uchicago.edu/persp012\_regression\_diagnostics.html}

\item {} 
\sphinxurl{https://stats.stackexchange.com/questions/218156/what-are-some-of-the-most-common-misconceptions-about-linear-regression}

\item {} 
\sphinxurl{https://pdfs.semanticscholar.org/a410/ec58bd5ff80a4e7978955ab11d096af6d138.pdf}

\item {} 
\sphinxhref{https://web.ma.utexas.edu/users/mks/statmistakes/StatisticsMistakes.html}{COMMON MISTEAKS MISTAKES IN USING STATISTICS: Spotting and Avoiding
Them}

\item {} 
\sphinxurl{http://www2.sas.com/proceedings/sugi22/STATS/PAPER267.PDF}

\item {} 
\sphinxurl{https://towardsdatascience.com/simple-linear-regression-2421076a5892}

\item {} 
\sphinxurl{http://jeromyanglim.blogspot.com/2013/12/using-r-to-replicate-common-spss.html}

\item {} 
\sphinxurl{https://stat.hevra.haifa.ac.il/~gweiss/courses/Forecasting/CarPricesAnalysis.pdf}

\item {} 
\sphinxurl{http://www.nrcresearchpress.com/doi/10.1139/f73-072\#.W-SvWuK\_Pcs}

\item {} 
\sphinxurl{https://www.nejm.org/doi/full/10.1056/NEJM198512263132604}

\item {} 
\sphinxurl{https://www.ncbi.nlm.nih.gov/pmc/articles/PMC2992018/} - medicine,
nice!

\item {} 
\sphinxurl{http://replication.uni-goettingen.de/wiki/index.php/Ordinary\_least\_squares\_(OLS})
- list of replications

\item {} 
\sphinxurl{https://stats.stackexchange.com/questions/211707/replicating-a-linear-regression-example-from-hastie-tibshirani-and-friedman}
(replication from Hastie)

\end{itemize}

Intuitive learning for OLS - \sphinxhref{https://stats.stackexchange.com/questions/tagged/linear-model?sort=votes\&pageSize=15}{Stackoverflow hot
topics}
- \sphinxhref{https://jakevdp.github.io/PythonDataScienceHandbook/05.06-linear-regression.html}{In Depth: Linear
Regression}
- Stachurski math intro: -
\sphinxhref{https://mitpress.mit.edu/sites/default/files/Stachurski\_final\_TOC.pdf}{TOC}
- \sphinxhref{https://github.com/jstac/econometric\_theory\_slides}{slides} -
\sphinxurl{https://lectures.quantecon.org/py/ols.html} -
\sphinxurl{http://www.statsmodels.org/stable/examples/index.html\#regression} -
\sphinxurl{https://docs.scipy.org/doc/scipy/reference/generated/scipy.linalg.lstsq.html}
-
\sphinxurl{https://stats.stackexchange.com/questions/1829/what-algorithm-is-used-in-linear-regression}


\subsection{Principal components analysis, PCA}
\label{\detokenize{techniques/pca::doc}}\label{\detokenize{techniques/pca:principal-components-analysis-pca}}
Math:
Assumptions:
Usual steps:
What may go wrong:
Discussion:
Replication examples:
\begin{itemize}
\item {} 
…

\end{itemize}

Links:
\begin{itemize}
\item {} 
\sphinxurl{https://stats.stackexchange.com/questions/2691/making-sense-of-principal-component-analysis-eigenvectors-eigenvalues/2700\#2700}

\item {} 
\sphinxurl{https://scikit-learn.org/stable/auto\_examples/decomposition/plot\_pca\_3d.html\#sphx-glr-auto-examples-decomposition-plot-pca-3d-py}

\item {} 
\sphinxurl{https://stats.stackexchange.com/questions/48214/replicating-shalizis-new-york-times-pca-example?rq=1}

\end{itemize}


\section{2.3 History of econometrics}
\label{\detokenize{topics:history-of-econometrics}}

\subsection{History of econometrics}
\label{\detokenize{history::doc}}\label{\detokenize{history:history-of-econometrics}}

\subsubsection{Landmark events}
\label{\detokenize{history:landmark-events}}\begin{itemize}
\item {} 
\sphinxhref{http://jse.amstat.org/v9n3/stanton.html}{Galton} and Pearson

\item {} 
Tinbergen, Haavelmo

\item {} 
Cowles Commission

\item {} 
Lucas critique

\end{itemize}


\subsubsection{Overview publications}
\label{\detokenize{history:overview-publications}}\begin{itemize}
\item {} 
\sphinxhref{http://ftp.iza.org/dp2458.pdf}{Econometrics: A Bird’s Eye View}

\item {} 
\sphinxhref{https://www.le.ac.uk/economics/research/RePEc/lec/leecon/dp14-05.pdf}{Econometrics: An Historical Guide for the Uninitiated}

\item {} 
\sphinxhref{https://www.researchgate.net/publication/24119912\_The\_First\_Fifty\_Years\_of\_Modern\_Econometrics}{The First Fifty Years of Modern Econometrics}

\end{itemize}


\subsubsection{By topic}
\label{\detokenize{history:by-topic}}\begin{itemize}
\item {} 
\sphinxhref{ucla.in/2mhxKdO}{Pearl, J. (2014). TRYGVE HAAVELMO AND THE EMERGENCE OF CAUSAL CALCULUS. Econometric Theory, 31(1), 152\textendash{}179. https://doi.org/10.1017/s0266466614000231}

\item {} 
\sphinxhref{https://arxiv.org/pdf/0808.2902.pdf}{A Short History of Markov Chain Monte Carlo (arxiv)}

\end{itemize}


\chapter{3. Applications}
\label{\detokenize{applications::doc}}\label{\detokenize{applications:applications}}
In this chapter we collect articles and chapters from several
practical domains (macro, finance, risk and microeconomics)
for study and replication.

Macro:
\begin{itemize}
\item {} 
GDP

\item {} 
FX

\item {} 
interest rates

\item {} 
asset class allocation

\item {} 
business cycle

\end{itemize}

Risks and finance:
\begin{itemize}
\item {} 
VAR

\item {} 
search of invariants (Atillo Meucci)

\end{itemize}

Panel data:
\begin{itemize}
\item {} 
\sphinxstyleemphasis{MHE}

\item {} 
policy assessment

\end{itemize}


\section{Finance}
\label{\detokenize{applications:finance}}

\subsection{Exchange rates}
\label{\detokenize{applications:exchange-rates}}\begin{itemize}
\item {} 
Messe/Rogoff

\item {} 
review by Barabara Rossi

\item {} 
anything by Sarno

\end{itemize}


\subsection{Interest rates}
\label{\detokenize{applications:interest-rates}}\begin{itemize}
\item {} 
Nelson-Siegel

\end{itemize}


\subsection{Value-at-risk}
\label{\detokenize{applications:value-at-risk}}\begin{itemize}
\item {} 
There must be something easy and approachable

\end{itemize}


\subsection{Credit risk, IFRS}
\label{\detokenize{applications:credit-risk-ifrs}}\begin{itemize}
\item {} 
There must be something easy and approachable

\item {} 
For now \sphinxhref{https://www.mathworks.com/campaigns/offers/ifrs-9-cecl-model-regulations-compliance.html}{this}

\end{itemize}


\section{Macroeconometrics}
\label{\detokenize{applications:macroeconometrics}}\begin{itemize}
\item {} 
Kalman and other time series filtering, state space representation

\item {} 
something on probability of recession from G Perez-Quiros

\item {} 
stationarity of macro time series

\item {} 
does Taylor rule work?

\item {} 
Lucas critique (?)

\item {} 
anything friendlier than Smets/Vouters for DSGE

\item {} 
\sphinxhref{https://www.mathworks.com/videos/matlab-and-macroeconomic-stress-testing-118689.html}{matlab-and-macroeconomic-stress-testing}

\end{itemize}


\section{Power sector}
\label{\detokenize{applications:power-sector}}

\subsection{Electricity prices}
\label{\detokenize{applications:electricity-prices}}\begin{itemize}
\item {} 
Nektaria’s and Derek’s paper

\end{itemize}


\section{Rest of JEL}
\label{\detokenize{applications:rest-of-jel}}
I know there are great replicable things throughout the JEL classification. I just do not know too much about them.


\subsection{Policy evaluation}
\label{\detokenize{applications:policy-evaluation}}
Anything they teach at \sphinxhref{https://irvapp.fbk.eu/trainings/detail/14117/irvapp-winter-school-2019-fundamentals-and-methods-for-impact-evaluation-of-public-policies-2019/}{irvapp}, \sphinxstyleemphasis{a variety of statistical tools for counterfactual analysis (including Matching methods, Instrumental Variables, Regression Discontinuity Designs, Difference-in-Differences, and Synthetic Control Methods)}


\chapter{References}
\label{\detokenize{references::doc}}\label{\detokenize{references:references}}

\section{Textbooks}
\label{\detokenize{references:textbooks}}
Favorite introductory:

\sphinxhref{https://scholar.google.com/scholar?cluster=16057448950849214316\&hl=en\&as\_sdt=0,5\&sciodt=0,5}{Peter Kennedy. A Guide To Econometrics}

General:
\begin{itemize}
\item {} 
Greene?

\item {} 
Dougherty?

\item {} 
Stock and Watson?

\item {} 
Verbeek?
\begin{itemize}
\item {} 
\sphinxhref{https://www.ssc.wisc.edu/~bhansen/econometrics/Econometrics.pdf}{Hansen. Econometircs}

\end{itemize}

\end{itemize}

\sphinxhref{https://scholar.google.com/scholar?q=related:3Mhz3KVE8GcJ:scholar.google.com/\&scioq=\&hl=en\&as\_sdt=0,5}{Time series}:
\begin{itemize}
\item {} 
\sphinxhref{https://scholar.google.com/scholar?cluster=7489561659476003036\&hl=en\&as\_sdt=0,5}{Hamilton}

\item {} 
\sphinxhref{https://www.sas.upenn.edu/~fdiebold/Teaching706/TimeSeriesEconometrics.pdf}{Diebold}

\item {} 
\sphinxhref{https://www.sciencedirect.com/science/article/pii/S1573441284020092}{Granger/Watson time series chapter in Handbook of Econometrics}

\end{itemize}

Machine learning:
\begin{itemize}
\item {} 
An Introduction to Statistical Learning. Gareth James, Daniela Witten, Trevor Hastie and Robert Tibshirani

\item {} 
Andrew Ng course

\end{itemize}

Deep learning:
\begin{itemize}
\item {} 
Deep Learning. Ian Goodfellow and Yoshua Bengio and Aaron Courville

\item {} 
fast.ai

\end{itemize}

Mention:
\begin{itemize}
\item {} 
\sphinxhref{http://cds.cern.ch/record/1320136/files/9780596529321\_TOC.pdf}{Toby Segaran. Programming Collective Intelligence. Building Smart Web 2.0 Applications}, also \sphinxhref{http://axon.cs.byu.edu/~martinez/classes/778/Papers/GP.pdf}{here}

\item {} 
\sphinxhref{http://www3.wabash.edu/econometrics/index.htm}{Excel-based}

\end{itemize}

Threads about picking a textbook:
\begin{itemize}
\item {} 
\sphinxurl{http://www.urch.com/forums/phd-economics/127240-just-finished-intro-wooldridge-gurajati-whats-best-text-next-self-study.html}

\end{itemize}


\section{Papers}
\label{\detokenize{references:papers}}
We choose papers that are:
\begin{itemize}
\item {} 
important milestones, ‘seminal’, widely cited, or

\item {} 
good subject area reviews, or

\item {} 
easy to understand and replicate, honest and friendly to reader, or

\item {} 
already have code and data attached,

\item {} 
or sevral things from the above.

\end{itemize}


\subsection{List of lists}
\label{\detokenize{references:list-of-lists}}\begin{itemize}
\item {} 
\sphinxurl{https://andrewgelman.com/2014/03/31/cited-statistics-papers-ever/}

\item {} 
\sphinxurl{https://core.ac.uk/download/pdf/6237204.pdf}

\item {} 
See JEL section C in \sphinxurl{https://hal.archives-ouvertes.fr/hal-01634432/document}

\item {} 
\sphinxurl{https://davegiles.blogspot.com/2012/06/highly-cited-statistical-papers-for.html}

\end{itemize}


\subsection{Seminal}
\label{\detokenize{references:seminal}}\begin{itemize}
\item {} 
Marquardt, Donald W. “An algorithm for least-squares estimation of nonlinear parameters.” Journal of the Society for Industrial \& Applied Mathematics 11.2 (1963): 431-441.

\item {} 
Granger, C.W.J. and P. Newbold, (1974), Spurious Regressions in Econometrics, Journal of Econometrics 2, 111\textendash{}120.

\item {} 
Akaike  H.  A  new  look  at  the  statistical  model  identification.  IEEE  Trans.  Automat.  Contr. AC-19:716-23,  1974.

\item {} 
Bollerslev, Tim. “Generalized autoregressive conditional heteroskedasticity.” Journal of econometrics 31.3 (1986): 307-327.

\item {} 
Engle, Robert F., and Clive WJ Granger. “Co-integration and error correction: representation, estimation, and testing.” Econometrica: journal of the Econometric Society (1987): 251-276.

\end{itemize}


\subsection{Popular}
\label{\detokenize{references:popular}}\begin{itemize}
\item {} 
\sphinxhref{http://www.columbia.edu/~jb3064/papers/2004\_A\_panic\_attack\_on\_unit\_roots\_and\_cointegration.pdf}{A PANIC ATTACK ON UNIT ROOTS AND COINTEGRATION Econometrica , Vol. 72, No. 4 (July, 2004), 1127\textendash{}1177}

\item {} 
\sphinxhref{https://scholar.harvard.edu/files/stock/files/efficient\_tests\_for\_an\_autoregressive\_unit\_root.pdf}{EFFICIENT TESTS FOR AN AUTOREGRESSIVE UNIT ROOT Econometrica, Vol. 64, No. 4 (July, 1996), 813-836}

\end{itemize}


\subsection{Domain knowledge}
\label{\detokenize{references:domain-knowledge}}\begin{itemize}
\item {} 
\sphinxhref{https://www.aeaweb.org/articles?id=10.1257/mac.1.2.189}{A Century of Work and Leisure}

\item {} 
\sphinxhref{https://www.nber.org/papers/w13943}{A Black Swan in the Money Market} - with simple OLS

\item {} 
\sphinxurl{https://economics.stackexchange.com/questions/10358/what-are-main-methods-for-econometrics-of-macroeconomics?rq=1}

\end{itemize}


\section{Blogs}
\label{\detokenize{references:blogs}}
The three blogs below are alone good to enlight and inspire about
statistics and econometrics. They are regularly updated.
\begin{itemize}
\item {} 
\sphinxhref{http://econometricsense.blogspot.com/}{Matt Bogard}

\item {} 
\sphinxhref{https://fxdiebold.blogspot.com/}{Francis Diebold}

\item {} 
\sphinxhref{https://davegiles.blogspot.com/}{Dave Giles}

\end{itemize}

\sphinxhref{https://simplystatistics.org/}{Simply Statistics} by biostatistics
professors Rafa Irizarry, Roger Peng, and Jeff Leek cover data management,
analysis and teaching stats.


\section{Course syllabi}
\label{\detokenize{references:course-syllabi}}
So many of them… just a few:
\begin{itemize}
\item {} 
\sphinxurl{http://www.sci.csueastbay.edu/~esuess/stat6620}

\item {} 
\sphinxurl{http://fmwww.bc.edu/EC-C/S2013/823/EC823.S2013.php}

\item {} 
\sphinxhref{https://www.schmidheiny.name/teaching/shortguides.htm}{Short guides in microeconometrics}

\item {} 
\sphinxurl{https://github.com/bdemeshev/em301/wiki/dream\_econometrics}

\item {} 
\sphinxurl{http://bdemeshev.github.io/em301/}

\item {} 
\sphinxurl{https://www.tristanfletcher.co.uk/tutorials/}

\end{itemize}


\section{Software user guides}
\label{\detokenize{references:software-user-guides}}\begin{itemize}
\item {} 
\sphinxhref{http://gretl.sourceforge.net/gretl-help/gretl-guide.pdf}{gretl}

\item {} 
\sphinxhref{http://www.math.unipd.it/~aiolli/corsi/1213/aa/user\_guide-0.12-git.pdf}{scilearn-kit dev}

\item {} 
statmodels

\item {} 
R

\end{itemize}


\section{Videos}
\label{\detokenize{references:videos}}\begin{itemize}
\item {} 
\sphinxhref{https://youtu.be/RpAF8Px-E\_A}{Gita Gopinath. What Have Economists Learned About the International Economy?}

\item {} 
\sphinxhref{https://www.analyticbridge.datasciencecentral.com/profiles/blogs/how-bayesian-inference-works-tutorial}{How Bayesian inference works}

\item {} 
\sphinxhref{https://t.co/dKNP3c5jeL}{Understanding Generalised Method of Moments}

\end{itemize}


\section{Other}
\label{\detokenize{references:other}}\begin{itemize}
\item {} 
\sphinxurl{https://radiant-rstats.github.io/docs/install.html} (app)

\item {} 
\sphinxhref{http://dagitty.net/primer}{Causal Inference In Statistics: A Companion for R Users}

\item {} 
\sphinxhref{http://www.stat.columbia.edu/~gelman/book/}{77 best lines from Andrew Gelman Bayesian extimation course}

\item {} 
\sphinxurl{http://petersonbiology.com/math230Notes/bootstrapSimulation.html}

\end{itemize}


\chapter{Data}
\label{\detokenize{data::doc}}\label{\detokenize{data:data}}\begin{itemize}
\item {} 
Python libraries \sphinxhref{https://kolesnikov.ga/Datasets\_in\_Python/}{statsmodels, sci-learn and seaborn}
and \sphinxhref{https://stat.ethz.ch/R-manual/R-devel/library/datasets/html/00Index.html}{R itself} have
build-in datasets.

\item {} 
Well known dataset repository is \sphinxhref{http://archive.ics.uci.edu/ml/datasets.html}{UCI}.

\item {} 
\sphinxhref{https://www.kaggle.com/datasets}{Kaggle} obviously has plenty of datasets.

\item {} 
Sheffield University made a good  listing of \sphinxhref{https://www.sheffield.ac.uk/mash/data}{datasets for teaching}.

\item {} 
Australian National Centre for Econometric Research website has
a useful \sphinxhref{http://www.ncer.edu.au/resources/data-and-code.php}{Data and code}
section.

\end{itemize}


\chapter{Good clues from Twitter}
\label{\detokenize{tweets::doc}}\label{\detokenize{tweets:good-clues-from-twitter}}
A secret subtitle for this publication is
“Can you learn econometrics from Twitter and Stack Overflow alone
without distracting yourself to machine-learning tutorials”.

Links collcted in no particular order, but will show in other sections of the Navigator.


\section{No links between leading macrotextbooks}
\label{\detokenize{tweets:no-links-between-leading-macrotextbooks}}



\section{OLS interactively exposed}
\label{\detokenize{tweets:ols-interactively-exposed}}



\section{R language guide is an econometrics guide}
\label{\detokenize{tweets:r-language-guide-is-an-econometrics-guide}}



\section{OLS, MML, Bayes and MCMC(!) for linear regression}
\label{\detokenize{tweets:ols-mml-bayes-and-mcmc-for-linear-regression}}
\sphinxurl{https://peterroelants.github.io/posts/linear-regression-four-ways/}


\section{Very true on R2}
\label{\detokenize{tweets:very-true-on-r2}}



\section{Instrumental variables}
\label{\detokenize{tweets:instrumental-variables}}



\section{Interpreting coefficients 1}
\label{\detokenize{tweets:interpreting-coefficients-1}}



\section{Interpreting coefficients 2}
\label{\detokenize{tweets:interpreting-coefficients-2}}



\section{Doing PCA approach}
\label{\detokenize{tweets:doing-pca-approach}}



\section{OLS explained for social scientists}
\label{\detokenize{tweets:ols-explained-for-social-scientists}}



\section{Suggested stat excercises}
\label{\detokenize{tweets:suggested-stat-excercises}}



\section{Coding a tree}
\label{\detokenize{tweets:coding-a-tree}}



\section{Consumer demand modelling}
\label{\detokenize{tweets:consumer-demand-modelling}}



\section{Scott Cameron learning method}
\label{\detokenize{tweets:scott-cameron-learning-method}}



\section{Model evaluation compendium (on classifier)}
\label{\detokenize{tweets:model-evaluation-compendium-on-classifier}}



\section{Causality by Judea Pearl}
\label{\detokenize{tweets:causality-by-judea-pearl}}



\section{A thesis turned tutorial on probabilistic programming and MC inference by Tom Rainforth}
\label{\detokenize{tweets:a-thesis-turned-tutorial-on-probabilistic-programming-and-mc-inference-by-tom-rainforth}}



\section{Value of logit}
\label{\detokenize{tweets:value-of-logit}}




\renewcommand{\indexname}{Index}
\printindex
\end{document}